\documentclass[10pt,a4paper]{article}
\usepackage[top=50pt,bottom=40pt,left=60pt,right=50pt]{geometry}
\usepackage[utf8]{inputenc}
\usepackage{amsmath}
\usepackage{amsfonts}
\usepackage{amssymb}
\usepackage{rotating}
\usepackage{verbatim}

\author{Judith van Stegeren and Mirjam van Nahmen}
\title{Development of the Mars Rover robot}

\begin{document}

\maketitle

%FirstPart:
 
% Description of the work on the Mars Rover
%   Describe the details of the main development activities:
\section*{Description of the work on the Mars Rover}
The main goal of the robot is to find all three colored lakes and measure the temperature.

% Listing of requirements with priorities.
\subsection*{MoSCoW overview of requirements}
\begin{tabular}{|p{5cm}|p{12cm}|} 
\hline
MoSCoW & requirement\\
\hline
\hline
  \textbf{Functional requirements} & The robot must be able to ...\\
\hline
M & find all lakes.\\
M & find a lake and determine its color.\\
M & navigate around lakes.\\
M & avoid falling off the table.\\
M & notice obstacles in the area and avoid them.\\
M & measure the temperature of all lakes.\\
M & see the difference between black, white, red, green and blue.\\
S & recognize bumping against an obstacles and drive around it.\\
S & show the overview with the information of all lakes.\\
\hline

  \textbf{Usability} & The robot must ...\\
\hline
M & be programmed by stating its behavior in a DSL and then generating code from that specification.\\
S & not drive faster then 0,20 m/s , so that if the robot is heading for the edge of the table, the user still has time to interfere.\\
S & provide feedback for the user about the state of the program for debugging purposes.\\
C & provide feedback for the user about measurements it makes.\\
M & stop when the robot is finished.\\
M & provide feedback for the user when the robot is finished and the user press a button.\\
C & provide feedback for the user about errors/bugs.\\
\hline

  \textbf{Reliability}& \\
\hline
M & The robot components must be tested before it will work with the whole system.\\
S & The robot must stop if an error/bug is detected to avoid to go off of the table. \\
S & The sensors must be calibrated before the program starts.\\
M & The robot must have enough power and memory to run the programs.\\
C & The robot must recognize its starting position.\\

\hline

  \textbf{Performance}&\\
\hline
M & If the light sensors spot the white border, the robot must react immediately.\\
S & If the light sensors spot the color border of a lake, the robot must react immediately.\\
C & The robot should find all lakes in 10 minutes. \\
S & The robot should not measure the same lake twice.\\

\hline
   \textbf{Supportability}&\\
\hline
W & The DSL can also be used for the generation of code in another language or for use with a different API.\\
\hline
\end{tabular}

\pagebreak 
% Deployment with distribution of sensors, actuators and software components over NXT bricks; give both the own proposal and the chosen deployment, including motivation.
\section*{Proposal deployment}
List of the actuators:
\begin{itemize}
\item[2] motors
\item[1] motor for  the temperature sensor
\item[1] lamp
\end{itemize}

\noindent List of the sensors:
\begin{itemize}
\item[1] temperature sensor
\item[2] light sensors
\item[1] color sensor
\item[1] ultra sonar
\item[2] touch sensors
\end{itemize}

There can be 3 actuators and 4 sensors on 1 brick.
Since it is vital that the motor does not fall off the table, we want that the communication between the light sensors and the motors happens without bluetooth. Since collision with other objects is an important issue, it would seem logical to connect the touch sensors directly with the master brick as well.\\ 

Since the color sensor, the motor for the temperature sensor and the temperature sensor all have to do with the same functionality, it would seem logical to put these sensors and actuators on the same brick. Finally, as the measurement of the distance with the ultrasonic sensor is not crucial for the functionality of the robot, we suggest to put that sensor on the (slave) brick for the measurements as well.\\

\subsubsection*{Deployment diagram}
\begin{tabular}{|p{3cm}|p{6cm}|p{6cm}|} 
\hline
brick & NXT1 (master) & NXT2 (slave)\\
\hline
tasks & wandering, navigating, avoiding collisions & recognizing colours, spotting far-away objects, temperature sensor control, lamp\\\hline
actuators & motor 1, motor 2 & temperature sensor motor\\\hline
sensors & light sensor 1, light sensor 2, touch sensor 1, touch sensor 2 & ultra sonic sensor, temperature sensor, color sensor\\
\hline
\end{tabular}

\subsection*{Actual deployment diagram}

\begin{minipage}[t]{0.4\textwidth}

\textbf{Brick 1 (Master):} \\
crucial processes (wandering without \\collisions or falling off the table)

	\begin{tabular}{|c|l|}
	\hline
	A & Motor left\\
	B & Motor right\\
	C & Lamp yellow\\
	S1 & Light sensor left\\
	S2 & Light sensor right\\
	S3 & Touch sensor left\\
	S4 & Touch sensor right\\
	\hline
	\end{tabular}
	
\end{minipage}
\begin{minipage}[t]{0.2\textwidth}
	\begin{itemize}
	\item[ ]
	\end{itemize}
\end{minipage}
\begin{minipage}[t]{0.4\textwidth}

\textbf{Brick 2 (Slave):} \\
measurements (color of a lake, temperature and distance detection)

	\begin{tabular}{|c|l|}
	\hline
	A & Motor Temperature\\
	B & \\
	C & Lamp green\\
	S1 & Color sensor\\
	S2 & Ultrasonic sensor\\
	S3 & Temperature sensor\\
	S4 & \\
	\hline
	\end{tabular}
\end{minipage}

\subsection*{Use cases}
\begin{description}
\item[Wandering within white line] The robot drives in a straight line until it encounters a white line with one of its light sensors. If it encounters the white line, it will back up a bit, make a turn away from the white line and proceed in a straight line. For this use case, we need both motors and both the light sensors.
\item[New lake detection and measurements] On detection of a colored lake, the robot first checks if it hasn't encountered the lake before. If not, the robot positions itself before the lake. It then lowers the temperature sensor, measures the temperature and stores it together with the color of the lake. We need the color detector, the motor for the temperature sensor and the temperature sensor itself.
\item[Old lake detection] On detection of a colored lake, the robot checks if he encountered this lake before. If this is the case, the robot drives around the lake. For this we need the color detector and the two motors, but it doesn't matter whether the color sensor is on the same brick as both motors.
\item[Wandering without collisions] The robot drives in a straight line. If one of the bumpers touches an object, the robot backs up, makes a turn and proceeds in a straight line. For this we need the bumpers and both motors. Since reaction time is important in this use case, it would be best if the motors and the bumpers are connected to the same brick. The ultrasonic sensor can also play a role in this use case. The reaction time of the ultrasonic sensor is less important since it can spot obstacles from further away. %The touch sensors are for the situation that the sonar missed something, so it could be an advantage to put the sonar also on the brick of the motors. The communication with the sonar does not have to be instantaneous, so we could use bluetooth, but it seems not the best thing to do.
\item[Ready] If the robot has measured the temperature in a lake, it check if all lakes have been measured. If that is the case, it stops wandering and shows the table with the temperatures of all lakes on his display.\\
\end{description}

% Listing of the identified risks.
\section*{Risk analysis}

\begin{tabular}{|p{0.4cm}|p{1.5cm}|p{3cm}|p{0.4cm}|p{0.4cm}|p{1.3cm}|p{3.5cm}|p{3.5cm}|} 
\hline
  Id & Type & Description & \rotatebox{90} {Probability} & \rotatebox{90} {Severity} & Weight & Mitigation & Contingency\\
\hline
\hline
 1 & Technical risk & The computer with the installed software breaks down and we lose access to our work. & 3 & 4 & High & Push all work to the repository, regularly. & Find a replacement for the broken laptop or use one of the lab computers.\\
\hline
 2 & Organiza- tional risk & The robot is not available for testing. & 3 & 4 & High & Regulary test small pieces of code, so that debugging is not a lot of work. & Postpone testing until later or wait until the robot is free. Use the lab when when we know there won't be other students around (early morning).\\
\hline
 3  & Technical risk & The DSL does not fit with the implementation. & 3 & 2 & Medium & Change the DSL to a lower level of abstraction. & \\
\hline
 4 & Organiza- tional risk & One of our team members become ill. & 3 & 2 & Small & Healthy living & The other one continues working on the project and the ill person works as much as possible from home. Keep communicating via email/chat.\\
\hline
 5 & Require- ments risk & The requirements contains ideas which are not compatible with the idea of the DSL. & 2 & 2 & Medium &  & Change requirements until it is possible to implement.\\
\hline
 6 & Technical risk & Problems with the bluetooth communication between the two bricks & 5 & 3 & Medium & Test bluetooth functionality as soon as possible. & Ask for help from the teacher or other students.\\
\hline
 7 & Organiza- tional risk & Lab partners have different schedules which makes it hard to find time to work together & 5 & 3 & Medium & The lab partners don't plan anything on Wednesday afternoon, so that there is at least one moment in the week when they are both free to work on the project & Reschedule work or work together during the weekends.\\
 \hline
\end{tabular}

\section*{DSL of the mars rover}
The idea behind the DSL is a state machine. In the practice it is an often used system. It uses states and arrows to describe the transitions and conditions in which the robot can be. For this we decided not to use the behavior functions in the  generation code but to implement our own arbitrator system. 

\subsection*{Syntax} 
%  Describe the concrete syntax of the DSL; what is the main idea, what is the meaning of the language?
\subsection*{Instances for the Mars Rover}
%  Describe the instance(s) of the DSL that are used for the Mars Rover, explain the main idea.

\subsection*{Main structure of the generated code}
% Describe the main structure of the generated code.

\section*{Realization of the Mars Rover}
% Describe the order in which features of the Mars Rover have been implemented.
The first steps were to design and implement the basic structure of the DSL and the code generation. We decided to use the state diagram and designed the start/end state and the robot should be able to wander an area without falling off the table. 

\subsection*{Final result}
% Describe the final result: what has been implemented exactly, which requirements are satisfied, which requirements are not satisfied and explain why not. 


\end{document}
