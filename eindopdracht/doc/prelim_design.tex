\documentclass[10pt,a4paper]{article}
\usepackage[top=50pt,bottom=40pt,left=60pt,right=50pt]{geometry}
\usepackage[utf8]{inputenc}
\usepackage{amsmath}
\usepackage{amsfonts}
\usepackage{amssymb}
\usepackage{rotating}
\usepackage{verbatim}

\author{Judith van Stegeren and Mirjam van Nahmen}
\title{Development of the Mars Rover robot}

\begin{document}

\maketitle

\section*{Specifications}
The main goal of the robot is to find all three colored lakes and measure the temperature.

\subsection*{Functional requirements}
The robot must be able to ...
\begin{itemize} 
\item wander the area.
\item recognize a lake and determine its color.
\item navigate around lakes.
\item recognize the white barrier and not cross it.
\item notice obstacles in the area and avoid them while wandering.
\item lower the temperature sensor and measure the temperature in a lake.
\item store the temperature with the color of the lake.
\item see the difference between black, white, red, green and blue.
\item recognize bumping against an obstacles and drive around it.
\end{itemize}


\subsection*{Usability}
The robot must ...
\begin{itemize}
\item be programmed by stating its behavior in a DSL and then generating code from that specification.
\item not drive too fast, so that if the robot is heading for the edge of the table, the user still has time to interfere.
\item provide feedback for the user about measurements it makes.
\item provide feedback for the user about the state of the program for debugging purposes.
\item provide feedback for the user when the robot is finished (show the table with measurements).
\item provide feedback for the user about errors/bugs.
\end{itemize}

\subsection*{Reliability}
\begin{itemize}
\item The robot must stop if an error/bug is detected.
\item The robot components must be tested before it will work with the whole system.
\item The sensors must be calibrated before the program starts.
\end{itemize}

\subsection*{Performance}
\begin{itemize}
\item If the light sensors spot the white border, the robot must react immediately to avoid falling off the table.
\item If the light sensors spot the color border of a lake, the robot must react immediately to avoid falling in the lake.
\item All three lakes must be found within a reasonable amount of time.
\item If the robot has already measured a lake, he must skip the measurements if he encounters that same lake again.
\end{itemize}

\subsection*{Supportability}
\begin{itemize}
\item The DSL can also be used for the generation of code in another language or for use with a different API.
\end{itemize}

\subsection*{MoSCoW overview of requirements}
\begin{tabular}{|p{3cm}|p{12cm}|} 
\hline
MoSCoW & requirement\\
\hline
M & wander the area.\\
M & recognize a lake and determine its color\\
M & navigate around lakes.\\
M & recognize the white barrier and not cross it.\\
M & notice obstacles in the area and avoid them while wandering.\\
M & lower the temperature sensor and measure the temperature in a lake.\\
M & see the difference between black, white, red, green and blue.\\
S & recognize bumping against an obstacles and drive around it.\\
S & store the temperature with the color of the lake.\\
\hline
M & be programmed by stating its behavior in a DSL and then generating code from that specification.\\
S & not drive too fast, so that if the robot is heading for the edge of the table, the user still has time to interfere.\\
S & provide feedback for the user about the state of the program for debugging purposes.\\
C & provide feedback for the user about measurements it makes.\\
C & provide feedback for the user when the robot is finished (show the table with measurements).\\
C & provide feedback for the user about errors/bugs.\\
\hline
M & The robot components must be tested before it will work with the whole system.\\
S & The robot must stop if an error/bug is detected to avoid to go off of the table. \\
S & The sensors must be calibrated before the program starts.\\
\hline
M & If the light sensors spot the white border, the robot must react immediately to avoid falling off the table.\\
S & If the light sensors spot the color border of a lake, the robot must react immediately to avoid falling in the lake.\\
S & All three lakes must be found within a reasonable amount of time.\\
S & If the robot has already measured a lake, he must skip the measurements if he encounters that same lake again.\\
\hline
W & The DSL can also be used for the generation of code in another language or for use with a different API.\\
\hline
\end{tabular}


\section*{Proposal deployment}
List of the actuators:
\begin{itemize}
\item[2] motors
\item[1] motor for  the temperature sensor
\end{itemize}

\noindent List of the sensors:
\begin{itemize}
\item[1] lamp
\item[1] temperature sensor
\item[2] light sensors
\item[1] color sensor
\item[1] ultra sonar
\item[2] touch sensors
\end{itemize}

There can be 3 actuators and 4 sensors on 1 brick.
Since it is vital that the motor does not fall off the table, we want that the communication between the light sensors and the motors happens without bluetooth. Since collision with other objects is an important issue, it would seem logical to connect the touch sensors directly with the master brick as well.\\ 

Since the color sensor, the motor for the temperature sensor and the temperature sensor all have to do with the same functionality, it would seem logical to put these sensors and actuators on the same brick. Finally, as the measurement of the distance with the ultrasonic sensor is not crucial for the functionality of the robot, we suggest to put that sensor on the (slave) brick for the measurements as well.\\

\subsection*{Deployment diagram}
\begin{tabular}{|p{3cm}|p{6cm}|p{6cm}|} 
\hline
brick & NXT1 (master) & NXT2 (slave)\\
\hline
tasks & wandering, navigating, avoiding collisions & recognizing colours, spotting far-away objects, temperature sensor control\\\hline
actuators & motor 1, motor 2 & temperature sensor motor\\\hline
sensors & light sensor 1, light sensor 2, touch sensor 1, touch sensor 2 & ultra sonic sensor, temperature sensor\\
\hline
\end{tabular}

\begin{comment}
\subsection*{Brick 1 (Master):} crucial processes (wandering without collisions or falling off the table)
\begin{itemize}
\item[] Motor 1
\item[] Motor 2
\item[] Light sensor 1
\item[] Light sensor 2
\item[] Touch sensor 1
\item[] Touch sensor 2
\end{itemize}

\subsection*{Brick 2:} measurements (color of a lake, temperature and distance detection)
\begin{itemize}
\item[] Motor Temperature
\item[] Temperature sensor
\item[] Ultrasonic sensor
\item[] Color sensor
\end{itemize}
\end{comment}

\subsection*{Use cases}
\begin{description}
\item[Wandering within white line] The robot drives in a straight line until it encounters a white line with one of its light sensors. If it encounters the white line, it will back up a bit, make a turn away from the white line and proceed in a straight line. For this use case, we need both motors and both the light sensors.
\item[New lake detection and measurements] On detection of a colored lake, the robot first checks if it hasn't encountered the lake before. If not, the robot positions itself before the lake. It then lowers the temperature sensor, measures the temperature and stores it together with the color of the lake. We need the color detector, the motor for the temperature sensor and the temperature sensor itself.
\item[Old lake detection] On detection of a colored lake, the robot checks if he encountered this lake before. If this is the case, the robot drives around the lake. For this we need the color detector and the two motors, but it doesn't matter whether the color sensor is on the same brick as both motors.
\item[Wandering without collisions] The robot drives in a straight line. If one of the bumpers touches an object, the robot backs up, makes a turn and proceeds in a straight line. For this we need the bumpers and both motors. Since reaction time is important in this use case, it would be best if the motors and the bumpers are connected to the same brick. The ultrasonic sensor can also play a role in this use case. The reaction time of the ultrasonic sensor is less important since it can spot obstacles from further away. %The touch sensors are for the situation that the sonar missed something, so it could be an advantage to put the sonar also on the brick of the motors. The communication with the sonar does not have to be instantaneous, so we could use bluetooth, but it seems not the best thing to do.
\item[Ready] If the robot has measured the temperature in a lake, it check if all lakes have been measured. If that is the case, it stops wandering and shows the table with the temperatures of all lakes on his display.\\
\end{description}

\section*{Risk analysis}

\begin{tabular}{|p{0.4cm}|p{1.5cm}|p{3cm}|p{0.4cm}|p{0.4cm}|p{1.3cm}|p{3.5cm}|p{3.5cm}|} 
\hline
  Id & Type & Description & \rotatebox{90} {Probability} & \rotatebox{90} {Severity} & Weight & Mitigation & Contingency\\
\hline
\hline
 1 & Technical risk & The computer with the installed software breaks down and we lose access to our work. & 3 & 4 & High & Push all work to the repository, regularly. & Find a replacement for the broken laptop or use one of the lab computers.\\
\hline
 2 & Organiza- tional risk & The robot is not available for testing. & 3 & 4 & High & Regulary test small pieces of code, so that debugging is not a lot of work. & Postpone testing until later or wait until the robot is free. Use the lab when when we know there won't be other students around (early morning).\\
\hline
 3  & Technical risk & The DSL does not fit with the implementation. & 3 & 2 & Medium & Change the DSL to a lower level of abstraction. & \\
\hline
 4 & Organiza- tional risk & One of our team members become ill. & 3 & 2 & Small & Healthy living & The other one continues working on the project and the ill person works as much as possible from home. Keep communicating via email/chat.\\
\hline
 5 & Require- ments risk & The requirements contains ideas which are not compatible with the idea of the DSL. & 2 & 2 & Medium &  & Change requirements until it is possible to implement.\\
\hline
 6 & Technical risk & Problems with the bluetooth communication between the two bricks & 5 & 3 & Medium & Test bluetooth functionality as soon as possible. & Ask for help from the teacher or other students.\\
\hline
 7 & Organiza- tional risk & Lab partners have different schedules which makes it hard to find time to work together & 5 & 3 & Medium & The lab partners don't plan anything on Wednesday afternoon, so that there is at least one moment in the week when they are both free to work on the project & Reschedule work or work together during the weekends.\\
 \hline
\end{tabular}


\end{document}
